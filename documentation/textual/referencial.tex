%Referencial Teórico

\chapter{Referência Teórica}

Neste capitulo será descrito quais as referências que incentivaram a escolha do tema em questão.

\section{Lógica Paraconsistente}

A LPA teve como precursores o lógico russo N. A. Vasiliev e o lógico polonês J.Lukasiewicz. Os dois em 1910, publicaram trabalhos independentes, porém se restringiam a lógica aristotélica tradicional. Entre 1948 e 1954 o lógico polonês S.Jaskowski e o lógico brasileiro Newton C.A. da Costa, independentes construíram a LPA \cite[p. 27]{tomda-decisao-lpa-2011}.

Segundo \citeonline{introd-lpa-2010} dentre as várias ideias no âmbito das Lógicas não-Clássicas criou-se uma família de lógicas que teve como fundamento principal a  revogação do princípio da Não Contradição, a qual foi nomeada de Lógica Paraconsistente. Portanto, a LPA é uma Lógica não-Clássica que
revoga o princípio da Não Contradição e admite o tratamento de informações
contraditórias na sua estrutura teórica.

\section{Teoria dos jogos}

\section{Biblioteca}

\section{Engenharia de Software}

Visando melhorar a qualidade dos produtos de software e aumentar a produtividade no processo de desenvolvimento, surgiu a Engenharia de Software. A Engenharia de Software trata de aspectos relacionados ao estabelecimento de processos, métodos, técnicas, ferramentas e ambientes de suporte ao desenvolvimento de software \cite[p. 2]{eng-2014}.

\subsection{Metodologia}

\subsection{UML}

Particularmente no tocante à engenharia de software, a \textit{Unified Modeling Language} (Linguagem de Modelagem Unificada - UML) pode ser utilizada para modelar todas as etapas do processo de desenvolvimento de software, bem como produzir todos os artefatos de Software necessários à documentação dessas etapas \cite[p. 12]{eng-2015}.

Segundo o mesmo autor (\citeyear{eng-2015}), a linguagem UML, por meio de seus diagramas, permite a definição e design de \textit{threads} (tarefas) e processos, que permitem o desenvolvimento de sistemas distribuídos ou de programação concorrente. Da mesma maneira, permite a utilização dos chamados \textit{patterns} são, a grosso modo, soluções de programação utilizadas devido ao seu bom desempenho e a descrição de colaborações esquemas de interação entre objetos que resultam em um comportamento do sistema.

\subsection{RUP}