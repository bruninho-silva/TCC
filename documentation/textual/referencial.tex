%Referencial Teórico

\chapter{Referência Teórica}

Neste capitulo serão apontados quais as referências que incentivaram a escolha do tema em questão.

\section{Lógica Paraconsistente}

A LPA teve como precursores o lógico russo N. A. Vasiliev e o lógico polonês J.Lukasiewicz. Os dois em 1910, publicaram trabalhos independentes, porém se restringiam a lógica aristotélica tradicional. Entre 1948 e 1954 o lógico polonês S.Jaskowski e o lógico brasileiro Newton C.A. da Costa, independentes construíram a LPA \cite[p. 27]{tomda-decisao-lpa-2011}.

Segundo \citeonline{introd-lpa-2010} dentre as várias ideias no âmbito das Lógicas não-Clássicas criou-se uma família de lógicas que teve como fundamento principal a  revogação do princípio da Não Contradição, a qual foi nomeada de Lógica Paraconsistente. Portanto, a LPA é uma Lógica não-Clássica que
revoga o princípio da Não Contradição e admite o tratamento de informações
contraditórias na sua estrutura teórica.

\section{Teoria dos jogos}

\section{Biblioteca}

A \textit{Library} ou Biblioteca que faz uso da Lógica Paraconsistente será criada, com intuito de ajudar ou facilitar a criação de jogos de carta mais especificamente para jogos TCG, a \textit{Library} vem com o objetivo de entregar funções já desenvolvida sem ter a necessidade de ter elaborar as funcionalidades do zero.
Com relação \textit{Application Programming Interface} (Interface de programação de aplicações - API) e um \textit{Framework} a uma \textit{Library}, elas podem ter uma
certa similaridade, mas tem conceitos distintos.

Biblioteca é uma conjunto de implementações de ações escritos em uma linguagem e importadas no seu código. Nesse caso, há uma interface bem definida para cada comportamento invocado.

A API tem um conceito diferente da Biblioteca, que basicamente contém um conjunto de instruções, rotina e padrões do código que contem acesso de um aplicativo especifico via conexão. Com tudo ele vem pra interpretar os dados e integra-los com outras plataformas e softwares, gerando instruções recém-lançadas que vem com intuito de executar por esses softwares \cite{api-2016}.

O \textit{Framework} é a base sólida e padronizada de uma aplicação resumidamente é a unificação de Bibliotecas e API’s de forma a oferecer uma estrutura ideal para desenvolver um aplicativo mobile \cite{abf-2016}.


\section{Engenharia de Software}

Visando melhorar a qualidade dos produtos de software e aumentar a produtividade no processo de desenvolvimento, surgiu a Engenharia de Software. A Engenharia de Software trata de aspectos relacionados ao estabelecimento de processos, métodos, técnicas, ferramentas e ambientes de suporte ao desenvolvimento de software \cite[p. 2]{eng-2014}.

\subsection{Metodologia}

\subsection{UML}

Particularmente no tocante à Engenharia de Software, a \textit{Unified Modeling Language} (Linguagem de Modelagem Unificada - UML) pode ser utilizada para modelar todas as etapas do processo de desenvolvimento de software, bem como produzir todos os artefatos de Software necessários à documentação dessas etapas \cite[p. 12]{eng-2015}.

Segundo o mesmo autor (\citeyear{eng-2015}), a linguagem UML, por meio de seus diagramas, permite a definição e design de \textit{threads} (tarefas) e processos, que permitem o desenvolvimento de sistemas distribuídos ou de programação concorrente. Da mesma maneira, permite a utilização dos chamados \textit{patterns} são, a grosso modo, soluções de programação utilizadas devido ao seu bom desempenho e a descrição de colaborações esquemas de interação entre objetos que resultam em um comportamento do sistema.

\subsection{RUP}