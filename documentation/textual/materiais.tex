% Materiais e Métodos

\chapter{Materiais e Métodos}

Para a elaboração do presente trabalho foram adotadas técnicas e metodologias que serão mostrados nos próximos tópicos.

\section{Linguagem C\#}

C\# ou \textit{C Sharp} é uma linguagem de programação desenvolvido pela Microsoft, que faz parte de sua plataforma \textit{.Net Framework}. Tem como aspecto o ser uma linguagem fortemente tipada que utiliza paradigma de orientação a objeto com sintaxe semelhante C,C++ ou JAVA. 

Segundo os autores \citeonline{Csharp-2002} Algumas das características essenciais do C\# que podem ser mencionadas:

\begin{itemize}
	\item Simplicidade: facilidade de codificação com alta performance;
	\item Completamente orientada a objetos: Tudo é um objeto em C\#;
	\item Fortemente tipada: atribuição de tipos para evitar manipulação imprópria ou incorreta;
	\item Flexibilidade: caso necessário C\# permitir o uso de ponteiros, ms com custo de desenvolver um código não gerenciado, chamado de \textit{unsafe};
	\item Linguagem  gerenciada: os programa desenvolvido em C\# é executado em ambiente gerenciado, na qual, o gerenciamento de memória é feito pelo \textit{Garbage Collector} (Coletor de lixo - GC);
\end{itemize}

A linguagem C\# é utilizada nesse projeto devido sua integração com a plataforma \textit{Unity}, na qual, será mais detalhada no próximo tópico.

\section{Plataforma Unity}

A plataforma Unity é conhecido como uma das melhores plataforma de desenvolvimento de jogos do mundo, justamente porque ela é potencializada em serviços e ferramentas sendo elas 2D e 3D.
Segundo \citeonline{unity-2019} \enquote{atualmente a plataforma domina 45\% do mercado global de desenvolvimentos de games, segundo a própria empresa; 34\% dos 1.000 maiores jogos mobile disponibilizado gratuitamente são feitos com a própria Unity.}

\section{Visual Studio}

