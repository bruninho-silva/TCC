% Materiais e Métodos

\chapter{Materiais e Métodos}

Para a elaboração do presente trabalho foram adotadas técnicas e metodologias que serão mostrados nos próximos tópicos.

\section{Linguagem C\#}

C\# ou \textit{C Sharp} é uma linguagem de programação desenvolvido pela Microsoft, que permite a criação de uma variedade de aplicativos executados no .NET Framework. Pode ser aplicada no desenvolvimento de aplicativos cliente-servidor, serviços Web XML, componentes distribuídos, aplicativos cliente-servidor, aplicativos de banco de dados entre outros.

Tem como aspecto ser uma linguagem fortemente tipada que utiliza o paradigma de orientação a objeto com sintaxe semelhante às linguagens C,C++ ou JAVA. Segundo os autores \citeonline{Csharp-2002} Algumas das características essenciais do C\# que podem ser mencionadas:

\begin{itemize}
	\item Simplicidade: facilidade de codificação com alta performance;
	\item Completamente orientada a objetos: Tudo é um objeto em C\#;
	\item Fortemente tipada: atribuição de tipos para evitar manipulação imprópria ou incorreta;
	\item Flexibilidade: caso necessário C\# permitir o uso de ponteiros, ms com custo de desenvolver um código não gerenciado, chamado de \textit{unsafe};
	\item Linguagem  gerenciada: os programa desenvolvido em C\# é executado em ambiente gerenciado, na qual, o gerenciamento de memória é feito pelo \textit{Garbage Collector} (Coletor de lixo - GC);
\end{itemize}

A linguagem C\# é utilizada nesse projeto devido sua integração com a plataforma \textit{Unity}, na qual, será mais detalhada no próximo tópico.

\section{Plataforma Unity}

A plataforma Unity é conhecido como uma das melhores plataforma de desenvolvimento de jogos do mundo, justamente porque ela é potencializada em serviços e ferramentas sendo elas 2D e 3D.
Segundo \citeonline{unity-2019} \enquote{atualmente a plataforma domina 45\% do mercado global de desenvolvimentos de games, segundo a própria empresa; 34\% dos 1.000 maiores jogos mobile disponibilizado gratuitamente são feitos com a própria Unity.}

\section{Visual Studio}

O Visual Studio é um  \textit{Integrated Development Environment} (Ambiente de desenvolvimento integrado - IDE) de \textit{open source} (código-fonte aberto) desenvolvido pela Microsoft, com recursos usados para auxiliar e simplificar no desenvolvimento de software.

De acordo com a \citeonline{microsoft-2019}, o software é composto por editor, compilador, ferramentas de preenchimento de código e designers gráficos, com o objetivo de facilitar a edição, criação, a depuração, o \textit{build} e a publicação do software desenvolvidos, além de possuir uma grande biblioteca de \textit{plugins} para interpretação de várias linguagens de programação com exemplo a linguagem Visual Basic, C, C++, C\#, F\#.

Terminado a apresentação do materiais e métodos em seguida e apresentado o capitulo \textit{4 Desenvolvimento.}
