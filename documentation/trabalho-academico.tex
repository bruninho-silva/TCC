% TCC

\documentclass[
	12pt,					% tamanho da fonte
	%openright,				% capítulos começam em pág ímpar (insere página vazia caso preciso)
	oneside,				% twoside para impressão em recto e verso. Oposto a oneside
	a4paper,				% tamanho do papel. 
	chapter=TITLE,			% títulos de capítulos convertidos em letras maiúsculas
	%section=TITLE,			% títulos de seções convertidos em letras maiúsculas
	%subsection=TITLE,		% títulos de subseções convertidos em letras maiúsculas
	%subsubsection=TITLE,	% títulos de subsubseções convertidos em letras maiúsculas
	english,				% idioma adicional para hifenização
	brazil					% o último idioma é o principal do documento
	]{abntex2}


% Pacotes básicos 
\usepackage{times}				% Usa a fonte times roman			
\usepackage[T1]{fontenc}		% Selecao de codigos de fonte.
\usepackage[utf8]{inputenc}		% Codificacao do documento (conversão automática dos acentos)
\usepackage{indentfirst}		% Indenta o primeiro parágrafo de cada seção.
\usepackage{color}				% Controle das cores
\usepackage{graphicx}			% Inclusão de gráficos
\usepackage{microtype} 			% para melhorias de justificação
\usepackage{multirow}			% Para mesclar linhas
\usepackage{amsmath}			%¨Módulo matemático
\usepackage{csquotes}			% Aspas

% Pacotes de citações
\usepackage[brazilian,hyperpageref]{backref}	% Paginas com as citações na bibl
\usepackage[alf]{abntex2cite}					% Citações padrão ABNT



% Configurações do pacote backref
% Usado sem a opção hyperpageref de backref
\renewcommand{\backrefpagesname}{Citado na(s) página(s):~}
% Texto padrão antes do número das páginas
\renewcommand{\backref}{}
% Define os textos da citação
\renewcommand*{\backrefalt}[4]{
	\ifcase #1 %
		Nenhuma citação no texto.%
	\or
		Citado na página #2.%
	\else
		Citado #1 vezes nas páginas #2.%
	\fi}%


% Informações de dados para CAPA e FOLHA DE ROSTO
\titulo{Biblioteca de Tomada de Decisão com lógica paraconsistente para jogos de cartas na unity}
\autor{
	Bruno de Paula Silva - C992534 \\
	Daniel Sousa David De Oliveira - D137GC0 \\
	Gustavo Felipe De Santana Marques - C993AH8 \\
	Marcelo Bueno Silva - N805CA0 \\
	Wesley Luiz Carvalho Silva - C993077}
\local{São Paulo}
\data{2019}
\orientador{Profª. Dr.a Amanda Luiza S. Pereira}
\instituicao{Universidade Paulista - UNIP}
\tipotrabalho{TCC}
\preambulo{Trabalho apresentado para aproveitamento da disciplina Trabalho de Curoso II , do curoso de Ciência da Computação, Da Universidade Paulista - UNIP Campus Cidade Universitária.}




% Configurações de aparência do PDF final
\definecolor{blue}{RGB}{0,0,0} % alterando o aspecto da cor azul

% informações do PDF
\makeatletter
\hypersetup{
     	%pagebackref=true,
		pdftitle={\@title}, 
		pdfauthor={\@author},
    	pdfsubject={\imprimirpreambulo},
	    pdfcreator={LaTeX with abnTeX2},
		pdfkeywords={abnt}{latex}{abntex}{abntex2}{trabalho acadêmico}, 
		colorlinks=true,       		% false: boxed links; true: colored links
    	linkcolor=blue,          	% color of internal links
    	citecolor=blue,        		% color of links to bibliography
    	filecolor=magenta,      	% color of file links
		urlcolor=blue,
		bookmarksdepth=4
}
\makeatother


% Posiciona figuras e tabelas no topo da página quando adicionadas sozinhas
% em um página em branco. Ver https://github.com/abntex/abntex2/issues/170
\makeatletter
\setlength{\@fptop}{5pt} % Set distance from top of page to first float
\makeatother



% Possibilita criação de Quadros e Lista de quadros.
% Ver https://github.com/abntex/abntex2/issues/176

\newcommand{\quadroname}{Quadro}
\newcommand{\listofquadrosname}{Lista de quadros}

\newfloat[chapter]{quadro}{loq}{\quadroname}
\newlistof{listofquadros}{loq}{\listofquadrosname}
\newlistentry{quadro}{loq}{0}

% configurações para atender às regras da ABNT
\setfloatadjustment{quadro}{\centering}
\counterwithout{quadro}{chapter}
\renewcommand{\cftquadroname}{\quadroname\space} 
\renewcommand*{\cftquadroaftersnum}{\hfill--\hfill}

\setfloatlocations{quadro}{hbtp} % Ver https://github.com/abntex/abntex2/issues/176

% Espaçamentos entre linhas e parágrafos 
\setlength{\parindent}{1.25cm} 	% O tamanho do parágrafo é dado por
\setlength{\parskip}{0.2cm} 	% Controle do espaçamento entre um parágrafo e outro


\makeindex 	% compila o indice


\begin{document} 		% Início do documento

\selectlanguage{brazil} % Seleciona o idioma do documento (conforme pacotes do babel)

\frenchspacing  		% Retira espaço extra obsoleto entre as frases.

% ----------------------------------------------------------
% ELEMENTOS PRÉ-TEXTUAIS
% ----------------------------------------------------------

\renewcommand{\imprimircapa}{
  \begin{capa}
    \center
    \ABNTEXchapterfont\Large Universidade Paulista - UNIP
    
    \vspace*{1cm}
    
    {\ABNTEXchapterfont\large\imprimirautor}

    \vfill
    \begin{center}
    \ABNTEXchapterfont\bfseries\LARGE\imprimirtitulo
    \end{center}
    \vfill
    
    \large\imprimirlocal

    \large\imprimirdata
    
    \vspace*{1cm}
  \end{capa}
}
\imprimircapa						% Inserir Capa

\include{pre-textual/folhaderosto} 				% Inserir Folha de rosto

% \include{pre-textual/fichacatalografica}		% Inserir a ficha bibliografica

% \include{pre-textual/errata} 					% Inserir errata

% \include{pre-textual/folhadeaprovacao} 		% Inserir folha de aprovação

% \include{pre-textual/dedicatoria} 			% Dedicatória

% \include{pre-textual/agradecimentos} 			% Agradecimentos

% \include{pre-textual/epigrafe} 				% Epígrafe

% \include{pre-textual/resumo} 					% Resumo em português e inglês

\include{pre-textual/listas} 					% Inseir listas

\include{sumario}								% inseir sumário


% ----------------------------------------------------------
% ELEMENTOS TEXTUAIS
% ----------------------------------------------------------
\textual

%Introdução

\chapter{Introdução}

Este trabalho acadêmico demonstra a aplicação da Lógica Paraconsistente Anotada (LPA), em jogos do gênero \textit{Tranding Card Games} (Jogos de Cartas Colecionáveis – TCG), através da criação de uma biblioteca de Tomada de Decisão que implementa a Lógica Paraconsistente Anotada Evidencial (LPA E$\tau$), além disso será criado um jogo de cartas que utiliza a biblioteca para demonstrar as suas funcionalidades.

Com parte do senso comum, as pessoas acreditam que os jogos têm como única finalidade entreter ignorando as diversas opções que um jogo eletrônico pode trazer para auxiliar o desenvolvimento humano, de modo que a utilização dos jogos de forma educacional ou para resolver problemas usando raciocínio lógico pode trazer benefícios à saúde \cite{fabio-luis-lpa}.

A LPA é uma lógica não clássica que admite contradições e incertezas, é uma boa solução para fazer tratamento de situações reais, no qual a Lógica Clássica, por ser binária, se mostra ineficaz ou impossibilitada de ser aplicada \cite{metodos-lpa-2006}. Assim possibilita as mais variadas aplicações em áreas tais como computação, robótica, tráfego aéreo e de trens, distribuição de energia em grandes usinas, programação, redes neurais, pesquisa operacional entre outras \cite{tomda-decisao-lpa-2011}.

Uma biblioteca é uma coleção de subprogramas ou um programa que facilita o desenvolvimento de sistemas, no núcleo da biblioteca desenvolvida será utilizado a
LPA. Dessa forma, a biblioteca implementada no jogo será responsável por tomar as decisões dos resultados de batalha, sendo o intuito de criar um software que pode ser reutilizável por outros, iniciando um estudo da aplicação da LPA em jogos TGC.

Será explicado como a biblioteca foi desenvolvida e implementada no jogo, juntamente com a sua documentação para utilização. Também será relatado como o
jogo foi desenvolvido, quais ferramentas e metodologias foram utilizadas e quais resultados que foram obtidos em vantagem com a utilização da LPA.

Este documento está estruturado nos seguintes tópicos, \textit{1 - Introdução} apresenta o projeto, os objetivos e as justificativas. No capítulo \textit{2 - Referência Teórica} é
exposta a base conceitual do projeto. Na seção seguinte \textit{3 - Materiais e Métodos} é retratada a metodologia utilizada para desenvolvimento da prototipagem além das ferramentas utilizadas no processo.

\pagebreak

\section{Justificativa}

No desenvolvimento de jogos de cartas, são encontrado diversas bibliotecas disponibilizado na \textit{Unity Asset Store}, com uso de LC, que uma proposição é classificada como verdadeira ou falsa. Não há qualquer outra possível alternativa, ou algo é Verdadeiro ou exclusivamente Falso \cite{aspectos-lpa-2013}.

A lógica paraconsistente introduz duas novas categorias além do Verdadeiro e do Falso. Podemos ter proposições classificadas como Verdadeiras, Falsas, Inconsistentes ou Paracompletas. Para uma proposição ser classificada com Inconsistente tem haver uma evidência sugere que ela seja Verdadeira e outra evidência aponte que ela é Falsa, Agora quando não tem evidência Verdadeira nem tampouco que ela seja Falsa a proposição é classificada como Paracompleta \cite{aspectos-lpa-2013}.

A proposta do projeto é desenvolver uma biblioteca aplicando TD com a LPA ao invés
do LC, para auxiliar no processo decisório aceitando valores contraditórios, possibilitando a criação de novas dinâmicas nos jogos.



\section{Validação Empírica}

A LPA é uma lógica não clássica que aceita contradições. A partir disso criar um cenário de jogo de cartas para demonstrar o uso da
paraconsistente. Para exemplificar, foram criadas quatro cartas com atributos de força e velocidade com valores favoráveis e desfavoráveis de acordo com arma e idade conforme a tabela \ref{tab:cartas}.

\begin{table}[htb]
	\centering
	\caption{Visualização das Cartas}
	\label{tab:cartas}
	\begin{tabular}{|l|l|l|l|l|l|}
		\hline
		\textbf{Carta}              & \textbf{Atributos}  & \textbf{Favorável} & \textbf{Desfavorável} & \textbf{Detalhes} & \textbf{Valor} \\ \hline
		\multirow{2}{*}{Arqueiro}   & \textit{Força}      & 20                 & 10                    & \textit{Arma}     & Arco e Flecha  \\ \cline{2-6} 
		& \textit{Velocidade} & 70                 & 35                    & \textit{Idade}    & 25             \\ \hline
		\multirow{2}{*}{Espadachim} & \textit{Força}      & 55                 & 20                    & \textit{Arma}     & Espada         \\ \cline{2-6} 
		& \textit{Velocidade} & 40                 & 20                    & \textit{Idade}    & 19             \\ \hline
		\multirow{2}{*}{Lanceiro}   & \textit{Força}      & 50                 & 10                    & \textit{Arma}     & Lança          \\ \cline{2-6} 
		& \textit{Velocidade} & 68                 & 35                    & \textit{Idade}    & 30             \\ \hline
		\multirow{2}{*}{Bárbaro}    & \textit{Força}      & 70                 & 50                    & \textit{Arma}     & Martelo        \\ \cline{2-6} 
		& \textit{Velocidade} & 60                 & 80                    & \textit{Idade}    & 40             \\ \hline
	\end{tabular}
	\fonte{Produzido pelos autores.}
\end{table}

O primeiro passo é realizar o processo de maximização, a partir do qual se obtém os maiores valores das evidências favoráveis e os menores das evidências
desfavoráveis, entre as cartas \textit{Arqueiro} e \textit{Espadachim}, repetindo o processo em relação às cartas \textit{Lanceiro} e \textit{Bárbaro}. Na sequência, realiza-se o processo de minimização, o qual consiste na obtenção dos menores valores das evidências favoráveis e dos maiores valores das evidências desfavoráveis, as quais foram maximizadas anteriormente.
Após realizar o processos de maximização e minimização nos dois atributos das cartas, obteve-se os seguintes valores:

\begin{figure}[htb]
	\caption{
		\label{fig:forca} 
		Valores
	}
	\begin{center}
		\includegraphics[scale=0.5]{imagens/valores.png}
	\end{center}
	\legend{Fonte: Produzido pelos autores.}

\end{figure}

Após a realização da maximização entre esses valores chegou-se ao seguinte resultado:

\begin{figure}[htb]
	\caption{
		\label{flg:max}
		Maximização
	}
	\begin{center}
		\includegraphics[scale=0.5]{imagens/max.png}
	\end{center}
	\legend{Fonte: Produzido pelos autores.}	
\end{figure}

Aplicando o grau de certeza e incerteza sobre esses valores a saída foi o seguinte estado lógico:

\begin{equation*}
	\begin{array}{cc}
	Gi = 0.68 + 0.1 -1 = -0.22 \\
	Gc = 0.68 - 0.1 = 0.58
	\end{array}
\end{equation*}

Através do estado lógico foi produzido o parecer analítico com uma tabela pré definida cujo resultado refere-se a porcentagem que as 4 cartas conseguem tirar de vida do adversário.

\newpage

\begin{table}[htb]
	\centering
	\caption{Relação entre o status e parecer analítico}
	\label{tab:analitica}
	\begin{tabular}{cc}
		\hline
		\multicolumn{1}{|c|}{\textbf{Status}} 	& \multicolumn{1}{c|}{\textbf{Parecer Analítico}} \\ \hline
		$V$               						& 100\%                      \\
		$F$               						& 0\%                        \\
		$T$               						& 0\%                        \\
		$\bot$									& 0\%                        \\
		$T \rightarrow V$             			& 20\%                       \\
		$T \rightarrow F$            			& 10\%                       \\
		$V \rightarrow \bot$            		& 16\%                       \\
		$F \rightarrow \bot$            		& 8\%                        \\
		$Qv \rightarrow T$            			& 50\%                       \\
		$Qv \rightarrow \bot$           		& 40\%                       \\
		$Qf \rightarrow T$            			& 6\%                        \\
		$Qf \rightarrow \bot$           		& 2\%                       
	\end{tabular}
	\fonte{Produzido pelos autores.}
\end{table}

O status obtido na maximização foi $T \rightarrow F$, conforme a tabela \ref{tab:analitica} a porcentagem seria de 10\%, com essa validação possuí a lógica que será aplicada para Tomada de Decisão no jogo, assim iniciar a criação da dinâmica do jogo.

\section{Objetivos}

O objetivo geral é desenvolver uma biblioteca de Tomada de Decisão que utilize a LPA com foco em jogos TCG com objetivo de, protótipo final, isso é, biblioteca com manual de utilização e jogo de demostração, com o intuito de ser utilizada por outros desenvolvedores de jogos de cartas


\section{Objetivos Específicos}
	\begin{itemize}
		\item Criar o modelo de paraconsistente a ser utilizado.
		\item Criar a biblioteca aplicando a LPA E$\tau$.
		\item Criar uma versão da biblioteca com valores fixo para uma implementação no jogo sem muitos problemas.
		\item Desenvolver uma segunda versão da biblioteca, permitindo que ela seja genérica o suficiente para atender diferentes regras de negócio em jogos TCG.
		\item Desenvolvimento da biblioteca.
		\item Construir um manual de utilização e documentação da biblioteca.
		\item Criar um jogo do gênero TCG.
		\item Implementar as funcionalidades da biblioteca no jogo.
		\item Gerar \textit{Asset} e disponibilizar na plataforma \textit{Unity Asset Store}.
		\item Criar a presente documentação descrevendo os materiais, métodos e referências utilizadas para a construção do projeto.
	\end{itemize}

	
Finalizada a apresentação da estrutura do trabalho, prossegue para \textit{capítulo 2 - Referência Teórica}.


%Referencial Teórico

\chapter{Referência Teórica}

Neste capitulo serão apontados quais as referências que incentivaram a escolha do tema em questão.

\section{Lógica Paraconsistente}

A LPA teve como precursores o lógico russo N. A. Vasiliev e o lógico polonês J.Lukasiewicz. Os dois em 1910, publicaram trabalhos independentes, porém se restringiam a lógica aristotélica tradicional. Entre 1948 e 1954 o lógico polonês S.Jaskowski e o lógico brasileiro Newton C.A. da Costa, independentes construíram a LPA \cite[p. 27]{tomda-decisao-lpa-2011}.

Segundo \citeonline{introd-lpa-2010} dentre as várias ideias no âmbito das Lógicas não-Clássicas criou-se uma família de lógicas que teve como fundamento principal a  revogação do princípio da Não Contradição, a qual foi nomeada de Lógica Paraconsistente. Portanto, a LPA é uma Lógica não-Clássica que
revoga o princípio da Não Contradição e admite o tratamento de informações
contraditórias na sua estrutura teórica.

\section{Teoria dos jogos}

A Teoria dos Jogos segundo \citeonline{sartini-tj-2004}, pode ser definida como a teoria dos modelos matemáticos que estuda a escolha de decisões ótimas sob condições de conflito.
E os elementos básicos dessa Teoria, segundo o mesmo, quando cada jogador escolhe sua estratégia, temos então uma situação ou perfil no espaço de todas as situações (perfis) possíveis. Cada jogador tem interesse ou preferências para cada situação no jogo.

As definições que constituem esta Teoria de acordo com \citeonline{cardoso-tj-2018}, visa compreender a racionalidade das decisões tomadas pelos jogadores, sempre com base na ideia de que impera a racionalidade na busca da melhor estratégia, ou seja, daquela que dará ao jogador a maior vantagem, seja na forma de mais lucro ou de mais satisfação.

\section{Biblioteca}

A Biblioteca fará o uso da LPA , com intuito de ajudar ou facilitar a criação de jogos de carta mais especificamente para jogos TCG, a \textit{Library} vem com o objetivo de entregar funções já desenvolvida sem ter a necessidade de ter que elaborar as funcionalidades do zero. Com relação a \textit{Application Programming Interface} (Interface de programação de aplicações - API) e um \textit{Framework} a uma \textit{Library} pode ter uma certa similaridade, mas tem conceitos distintos.

Biblioteca é uma conjunto de implementações de ações escritos em uma linguagem e importadas no seu código. Nesse caso, há uma interface bem definida para cada comportamento invocado.

A API tem um conceito diferente da Biblioteca, que basicamente contém um conjunto de instruções, rotina e padrões do código que Contém acesso de um aplicativo especifico via conexão. Com tudo ele vem pra interpretar os dados e integra-los com outras plataformas e softwares. \cite{api-2016}.

O \textit{Framework} é a base sólida e padronizada de uma aplicação resumidamente é a unificação de Bibliotecas e API’s de forma a oferecer uma estrutura ideal para desenvolver um software \cite{abf-2016}.


\section{Engenharia de Software}

Visando melhorar a qualidade dos produtos de software e aumentar a produtividade no processo de desenvolvimento, surgiu a Engenharia de Software. A Engenharia de Software trata de aspectos relacionados ao estabelecimento de processos, métodos, técnicas, ferramentas e ambientes de suporte ao desenvolvimento de software \cite[p. 2]{eng-2014}.

\subsection{Metodologia}

Metodologia é um conjunto de passos para alcançar um determinado objetivo. Em engenharia de software é conjunto de práticas que abrange todo o ciclo de vida do software que pode ser divida em três partes \cite{almeida-2017}:

\begin{itemize}
	\item Definição é etapa que feito o levantamento de informações, das funcionalidades desejadas, restrições e validação do projeto.
	\item Desenvolvimento é etapa de estruturação dos dados e planejamento de como fazer o software.
	\item Manutenção é etapa focada em correções de erros e melhorias no software.
\end{itemize}

\subsection{UML}

Particularmente no tocante à Engenharia de Software, a \textit{Unified Modeling Language} (Linguagem de Modelagem Unificada - UML) pode ser utilizada para modelar todas as etapas do processo de desenvolvimento de software, bem como produzir todos os artefatos de Software necessários à documentação dessas etapas \cite[p. 12]{eng-2015}.

Segundo o mesmo autor (\citeyear{eng-2015}), a linguagem UML, por meio de seus diagramas, permite a definição e design de \textit{threads} (tarefas) e processos, que permitem o desenvolvimento de sistemas distribuídos ou de programação concorrente. Da mesma maneira, permite a utilização dos chamados \textit{patterns} são, a grosso modo, soluções de programação utilizadas devido ao seu bom desempenho e a descrição de colaborações esquemas de interação entre objetos que resultam em um comportamento do sistema.

\subsection{RUP}

O Modelo \textit{rational unified process} ( Processo Unificado da Rational - RUP) foi criado pela \textit{Rational Software corporation} adquirida posteriormente pela \textit{International Business Machines} (IBM), que pode ser customizado de acordo com as necessidades do projeto, deixando assim, RUP mais leve (ágil) ou mais pesado (tradicional).

Conforme \citeonline{piske-2003} “[...] RUP é mais do que um softwares para auxiliar no desenvolvimento é uma metodologia de desenvolvimento, com uma estrutura formal e bem definida.”

Os ciclos de desenvolvimento são divididos em 4 etapas:


\begin{enumerate}
	\item Iniciação - levantamento de requisitos, funções desejáveis e criação do escopo do projeto.
	\item Elaboração -  analisar o escopo do projeto, estabelecer arquitetura, coletar os requisitos, desenvolver um plano para o projeto e mitigar os riscos do projeto.
	\item Construção - desenvolvimento e testes do software.
	\item Transição - validação e entrega do projeto.
\end{enumerate}

Além disso, o RUP possui nove disciplinas que contém \textit{templates} (modelos) que abrange todo o ciclo de vida de software, esses \textit{templates} ajudaram no planejamento do projeto, alguns modelos continham algumas perguntas que não se enquadraram no tamanho do projeto, porém o RUP é adaptável a diversos tipos de projetos tanto grandes quanto pequenos.  Além de após efetuar o preenchimento dos \textit{templates}, foi descoberto várias situações não previstas antes do preenchimento do RUP que puderam ser planejadas. Na seção de apêndices será apresentado alguns \textit{templates} do RUP preenchidos com base no projeto desenvolvido pelo grupo para conclusão de curso. 

Exposto as Referencial teórico do trabalho, segue para o capítulo \textit{3 - Materiais e métodos.}

% Materiais e Métodos

\chapter{Materiais e Métodos}

Para a elaboração do presente trabalho foram adotadas técnicas e metodologias que serão mostrados nos próximos tópicos.

\section{Linguagem C\#}

C\# ou \textit{C Sharp} é uma linguagem de programação desenvolvido pela Microsoft, que permite a criação de uma variedade de aplicativos executados no .NET Framework. Pode ser aplicada no desenvolvimento de aplicativos cliente-servidor, serviços Web XML, componentes distribuídos, aplicativos cliente-servidor, aplicativos de banco de dados entre outros.

Tem como aspecto ser uma linguagem fortemente tipada que utiliza o paradigma de orientação a objeto com sintaxe semelhante às linguagens C,C++ ou JAVA. Segundo os autores \citeonline{Csharp-2002} Algumas das características essenciais do C\# que podem ser mencionadas:

\begin{itemize}
	\item Simplicidade: facilidade de codificação com alta performance;
	\item Completamente orientada a objetos: Tudo é um objeto em C\#;
	\item Fortemente tipada: atribuição de tipos para evitar manipulação imprópria ou incorreta;
	\item Flexibilidade: caso necessário C\# permitir o uso de ponteiros, ms com custo de desenvolver um código não gerenciado, chamado de \textit{unsafe};
	\item Linguagem  gerenciada: os programa desenvolvido em C\# é executado em ambiente gerenciado, na qual, o gerenciamento de memória é feito pelo \textit{Garbage Collector} (Coletor de lixo - GC);
\end{itemize}

A linguagem C\# é utilizada nesse projeto devido sua integração com a plataforma \textit{Unity}, na qual, será mais detalhada no próximo tópico.

\section{Plataforma Unity}

A plataforma Unity é conhecido como uma das melhores plataforma de desenvolvimento de jogos do mundo, justamente porque ela é potencializada em serviços e ferramentas sendo elas 2D e 3D.
Segundo \citeonline{unity-2019} \enquote{atualmente a plataforma domina 45\% do mercado global de desenvolvimentos de games, segundo a própria empresa; 34\% dos 1.000 maiores jogos mobile disponibilizado gratuitamente são feitos com a própria Unity.}

\section{Visual Studio}

O Visual Studio é um  \textit{Integrated Development Environment} (Ambiente de desenvolvimento integrado - IDE) de \textit{open source} (código-fonte aberto) desenvolvido pela Microsoft, com recursos usados para auxiliar e simplificar no desenvolvimento de software.

De acordo com a \citeonline{microsoft-2019}, o software é composto por editor, compilador, ferramentas de preenchimento de código e designers gráficos, com o objetivo de facilitar a edição, criação, a depuração, o \textit{build} e a publicação do software desenvolvidos, além de possuir uma grande biblioteca de \textit{plugins} para interpretação de várias linguagens de programação com exemplo a linguagem Visual Basic, C, C++, C\#, F\#.

Terminado a apresentação do materiais e métodos em seguida e apresentado o capitulo \textit{4 Desenvolvimento.}


\phantompart

% ----------------------------------------------------------
% ELEMENTOS PÓS-TEXTUAIS
% ----------------------------------------------------------
\postextual



\bibliography{references} % Referências bibliográficas

% Inicia os apêndices
% \begin{apendicesenv}

% \partapendices % Imprime uma página indicando o início dos apêndices

% \chapter{rup}

% \chapter{documentação Biblioteca}

% \end{apendicesenv}


% INDICE REMISSIVO
\phantompart
\printindex

\end{document}
